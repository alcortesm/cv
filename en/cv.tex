%% start of file `moderncv_ntrp_template_en.tex'.
%% Copyright 2007 Xavier Danaux (xdanaux@gmail.com).
%
% This work may be distributed and/or modified under the
% conditions of the LaTeX Project Public License version 1.3c,
% available at http://www.latex-project.org/lppl/.
%
% Modded by ntrp (nitropowered@gmail.com)

\documentclass[11pt,a4paper]{moderncv}

% moderncv themes
%\moderncvtheme[blue]{casual}                 % optional argument are 'blue' (default), 'orange', 'red', 'green', 'grey' and 'roman' (for roman fonts, instead of sans serif fonts)
%\moderncvtheme[green]{classic}                % idem
\moderncvtheme[black]{classic}                % idem

\usepackage{CJKutf8}

\usepackage[T1]{fontenc}
% character encoding
\usepackage[utf8]{inputenc}                   % replace by the encoding you are using
\usepackage[english]{babel}

% adjust the page margins
\usepackage[scale=0.8]{geometry}
\recomputelengths                             % required when changes are made to page layout lengths

\usepackage{xcolor}

\fancyfoot{} % clear all footer fields
\fancyfoot[L,R]{\thepage}           % page number in "outer" position of footer line
\fancyfoot[R,L]{\footnotesize } % other info in "inner" position of footer line

% personal data
\firstname{Alberto}
\familyname{Cortés}
\title{Curriculum Vit\ae}               % optional, remove the line if not wanted
\email{alcortesm@gmail.com}                      % optional, remove the line if not wanted
\mobile{+34 600 42 77 57}                    % optional, remove the line if not wanted
\address{Madrid, Spain, UTC+1/+2}    % optional, remove the line if not wanted
%\phone{+34 ...}                      % optional, remove the line if not wanted
%\fax{<Fax number>}                          % optional, remove the line if not wanted
%\extrainfo{Skype: alberto.cortes.martin}
%\extrainfo{additional information (optional)} % optional, remove the line if not wanted
\photo[84pt]{../alcortes.jpg}                         % '64pt' is the height the picture must be resized to and 'picture' is the name of the picture file; optional, remove the line if not wanted

%\nopagenumbers{}                             % uncomment to suppress automatic page numbering for CVs longer than one page


%----------------------------------------------------------------------------------
%            content
%----------------------------------------------------------------------------------
\begin{document}
\maketitle

\definecolor{link}{RGB}{0,0,238} % standard blue link color
%Section
\section{General Information}
\cvline{Languages}{\small Spanish (native) \newline English (fluent, equivalent to CEFR C2 or Proficiency CPE)}
\cvline{GitHub}{\small \href{https://github.com/alcortesm}{\color{link}{https://github.com/alcortesm}} \normalsize}
\cvline{ResearcherId}{\href{http://www.researcherid.com/rid/L-5343-2014}{\color{link}{http://www.researcherid.com/rid/L-5343-2014}} \normalsize}
\cvline{LinkedIn}{\small \href{https://www.linkedin.com/in/alcortesm}{\color{link}{https://www.linkedin.com/in/alcortesm}} \normalsize}

%Section
\section{Desired Employment and Skills}

\cvline{}{\small
I am looking for a full remote job
as a Go programmer
writing distributed systems.
}

\cvline{}{\small
During my academic years,
I developed keen critical thinking skills
and a high understanding of the core, theoretical concepts of computation and networking;
I feel comfortable working with the mathematics of sophisticated data structures and algorithms.
}

\cvline{}{\small
As part of my current career path as a developer,
I have acquired the practical skills needed in a professional setting
and focus on domain design, maintainability, and keeping complexity under control.
}

%Section
\section{Computer Skills}
\cvline{Languages}{\small
  expert:
  Go (+10 years),
  Previously:
  C,
  Java,
  octave,
  \LaTeX,
  Node.js,
  C++,
  Python,
  Lua,
  Lisp,
  C\#,
  SR,
  Prolog.
}

\cvline{Tools and Technologies}{\small
  git (+internals),
  Kubernetes,
  Docker,
  Mongodb,
  Redis,
  Clouds (AWS, GCP),
  CI (Gitlab CI/CD, CircleCI, Github Actions...),
  OpenTelemetry (Prometheus, Grafana),
  S3, SQS, SNS, Google PubSub,
  protobuf,
  gRPC,
  REST,
  Istio,
  OAuth,
  make,
  Bazel,
  AMQP (RabittMQ),
  cryptography (SSH, IAM, JWT, Auth0, PGP),
  Couchbase,
  InfluxDB,
  GraphQL,
  PostgreSQL,
  pthreads,
  tcpdump,
  sockets,
  strace,
  valgrind,
  gdb,
  GnuPlot,
  gRPC,
  MessagePack,
  ns2,
  Omnet++,
  troff.
}

\cvline{OSs}{\small
  Linux (user-land and kernel).
  Plan9 (user-land and kernel),
  Windows XP,
  Symbian,
  Windows Mobile.
}

\cvline{Protocols}{\small
  gRPC,
  HTTP,
  TCP,
  IPv4,
  UDP.
  SCTP,
  POP3,
  IMAP,
  802.11b/g,
  Ethernet\ldots
}

\newpage

%Section
\section{Experience}

\cventry{2021-2025}
  {Back-End Senior Developer}
  {\href{https://www.faceit.com/}{\color{link}{Faceit}}, now \href{https://https://eslfaceitgroup.com/}{\color{link}{EFG}}}
  {full remote}
  {UK}
  {
  \begin{itemize}
%% Examples of projects:
% - write user-api, a port from a Java monolith to a Go microservice
% - understand, document and bring back under control two services written by an isolated developer now gone: matcher (matchmaking logic) and edge (WebSockets)
% - help the community team to refactor lobby-api using a domain driven approach, so they can introduce new features.
% - re-design user-api so Russian users Personal Identifiable Information can be stored in a Russian database to comply with Russian regulation
% - design proposal for using the outbox pattern
% - introduce strong domain design practices: optionals, immutable data types, making ilegal stats unrepresentable
% - introduce strong unit testing practices: idioms to avoid flaky tests and to improve readability (GIVEN-WHEN-THEN)
% - bring the drops system under control, automate many manual tasks, such as calculating drops per scheduled videos in Twitch
% - start using OAuth for authenticating users and third-party applications
% - propose an outbox pattern solution to ensure broker event delivery, using Debezium
% - interview candidates
    \item Architected and maintained distributed systems serving millions of users
    \item Developed greenfield, brownfield, and legacy systems
    \item Worked both in product teams and in the core platform team
    \item Owned and maintained critical shared libraries for core infrastructure components (databases, message brokers, observability, background processing)
    \item Established backend engineering standards and best practices for the backend guild (6 product teams $\approx$30 developers); facilitated guild meetings and mentored junior/mid-level engineers
    \item Maintained ownership of 2-6 production services
    \item 1-4 daily deployments to production
    \item Technical leadership focus: maintainability, testing, domain driven design and tech debt management
  \end{itemize}
}

\cventry{2020-2021}
  {Back-End Developer}
  {\href{https://www.couchbase.com/}{\color{link}{Couchbase}}}
  {full remote}
  {US, India, UK}
  {
  \begin{itemize}
    \item Cloud team: writing monitoring software for kubernetes clusters running Couchbase
  \end{itemize}
}

\cventry{2017-2020}
  {Back-End Developer}
  {\href{https://www.senseye.io}{\color{link}{Senseye}}}
  {full remote}
  {UK, US}
  {
  \begin{itemize}
    \item Condition monitoring and predictive maintenance for factory assets
    \item $\approx$ 100 services, $\approx$ 2-8 deploys to production per day
    \item Common tasks:
      adding new functionality
      and enabling the system to scale
    \item Personal focus: maintainability, testing and keeping tech debt under control
  \end{itemize}
}

\cventry{2015-2017}
  {Back-End Developer}
  {source\{d\}}{Madrid}{Spain}{
  \begin{itemize}
    \item Data processing pipeline for analysing git repositories
    \item A Git internals re-implementation in Go (\href{https://github.com/src-d/go-git}{\color{link}{go-git}})
    \item Abstract Syntax Trees generation and processing (\href{https://github.com/bblfsh}{\color{link}{bblfsh}})
    \item Talks and blog posts:
      \begin{itemize}
        \item Papers We Love Madrid meetup: \href{https://www.youtube.com/watch?v=P18BYZvnSx8}{\color{link}{Diff and Blame: an intuitive overview}}
        \item Comparing Merkle trees: \href{https://github.com/alcortesm/blog-post-difftree/blob/master/README.md}{\color{link}{Comparing Git trees in Go}}
      \end{itemize}
  \end{itemize}
}

\cventry{2004-2015}{Researcher and University Professor}{\href{https://www.uc3m.es}{\color{link}{UC3M}}}{Madrid}{Spain}{
  \begin{itemize}
    \item Teaching programming, OSs, concurrency and transport layer network protocols
    \item Experience writing technical documentation, presentations and public speaking
    \item Research projects:
        Measuring computational performance of mobile devices,
        Optimizing battery consumption of network protocols,
        Improving multihoming network protocols for streaming,
        Improving indoor localization techniques.
  \end{itemize}
} % arguments 3 to 6 are optional

%Section
\section{Education}
\cventry{2008-2012}{\small Ph.D. Telematics Engineering (Cum Laude)}{UC3M}{Madrid}{Spain}{\small \href{http://e-archivo.uc3m.es/bitstream/handle/10016/15108/Tesis_Alberto_Cortes_Martin.pdf?sequence=1}{\color{link}{My thesis}} (in Spanish); main contributions: An SCTP modification to improve handovers, a synthetic filesystem for the Linux kernel to improve multihoming support and a theorem for optimizing probe scheduling for the detection of uniform random events.\normalsize} % arguments 3 to 6 are optional
\cventry{2005-2007}{\small M.Sc. Telematics Engineering}{UC3M}{Madrid}{Spain}{} % arguments 3 to 6 are optional
\cventry{1995-2004}{\small B.Sc. Telecommunications Engineering (with Honours)}{UC3M}{Madrid}{Spain}{} % arguments 3 to 6 are optional

%Section
%\section{Master thesis}
%\cvline{title}{\emph{Title}}
%\cvline{supervisors}{Supervisors}
%\cvline{description}{\small Short thesis abstract}

%Section
\section{Interests and Hobbies}

\cvline{}{\small The Go board game (\begin{CJK}{UTF8}{min}囲碁\end{CJK},
  11kyu), free climbing, mini painting, Rubik's cubes, hacking, Science Fiction books, juggling.}

\closesection{}                   % needed to renewcommands
\renewcommand{\listitemsymbol}{-} % change the symbol for lists

% Publications from a BibTeX file
%\nocite{*}
%\bibliographystyle{plain}
%\bibliography{publications}       % 'publications' is the name of a BibTeX file

\end{document}
